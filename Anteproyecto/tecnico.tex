\documentclass[titlepage, 12pt, a4paper, oneside]{article}
\usepackage[utf8]{inputenc}
\usepackage[spanish, es-tabla]{babel}
\usepackage[T1]{fontenc}
\usepackage{hyperref}
\usepackage[numbib]{tocbibind}
\usepackage{tikz}
\usepackage[top=1in, bottom=1.25in, left=1.25in, right=1.25in]{geometry}
\usepackage{xcolor}

\usepackage{fancyhdr}
\pagestyle{fancy}
\fancyhf{}
\rhead{\textit{\color[rgb]{0.0,0.424,0.616}Nombre del estudiante}}
\lhead{}
\rfoot{}
\renewcommand{\headrulewidth}{0pt}

\usepackage{pgfgantt}

\title{}
\date{}
\renewcommand{\familydefault}{\sfdefault}

\begin{document}
\thispagestyle{empty}
\tikz[remember picture,overlay] \node[opacity=1.0,inner sep=0pt] at (current page.center){\includegraphics[width=\paperwidth,height=\paperheight]{Plantilla_AnteProyectoTFG-portada}};

\begin{center}
  \vspace{4cm}
  {\color{white} \Huge \textbf{Paralelismo de Datos en el Intercomunicador intercom-TM}}
\end{center}

\Large

\vspace{16.5ex}
\begin{tabular}{ll}
  ~~~~~~~~~ & Nombre del estudiante
\end{tabular}

\vspace{1.2cm}
\begin{tabular}{ll}
  ~~~~~~~~~ & Grado en Ingeniería Informática
\end{tabular}

\vspace{1.1cm}
\begin{tabular}{ll}
  ~~~~~~~~~ & Vicente González Ruiz
\end{tabular}

\vspace{1.2cm}
\begin{tabular}{ll}
  ~~~~~~~~~ & Departamento de Informática
\end{tabular}

\vspace{0.95cm}
\begin{tabular}{ll}
  ~~~~~~~~~ & Trabajo Técnico
\end{tabular}

\vspace{0.95cm}
\begin{tabular}{ll}
  ~~~~~~~~~ & Paralelismo de Datos, Audio, Intercomunicador, \\
  ~~~~~~~~~ & Procesamiento en Tiempo Real
\end{tabular}

\clearpage

\tikz[remember picture,overlay] \node[opacity=1.0,inner sep=0pt] at (current page.center){\includegraphics[width=\paperwidth,height=\paperheight]{Plantilla_AnteProyectoTFG-paginas}};

\normalsize

\section{Introducción}
Un intercomunicador (intercom en inglés), es un dispositivo que
permite a dos o más usuarios comunicarse en tiempo real. Normalmente
sólo transmiten sonido, aunque también pueden usar imágenes
(vídeos). Se usan en los portales de las casas para saber quién está
llamando en la puerta, lo usan los motoristas para hablar entre ellos
cuando se desplazan y (lógicamente) llevan cascos, y se usa en
aplicaciones de comunicación a través de Internet, como Skype,
WhatsApp, etc.

El intercomunicador intercom-TM es una aplicación desarrollada en la
Universidad de Almería por los alumnos de la asignatura de Tecnologías
Multimedia~\cite{intercom-TM}.

\section{Objetivos y justificación}
El objetivo de este TFG es mejorar la implementación actual de
intercom-TM, aprovechando el paralelismo de
datos~\cite{pacheco2011introduction}. Básicamente se trata de
aprovechar las posibilidades multicore de la mayoría de las
computadoras actuales para incrementar el desempeño de
intercom-TM. Más concretamente, los objetivos que se desean alcanzar
son:
\begin{enumerate}
\item Reimplementar intercom-TM usando el paradigma de programación de
  paralelismo de datos, que básicamente consiste en aplicar el
  algoritmo actualmente usado en la versión secuencial sobre bloques
  de datos, de forma concurrente. Esto permitirá aprovechar un número
  de cores de nuestro dispositivo (si los hubiera) superior a 1.
\item Determinar la carga de trabajo~\cite{workload} máxima asignada a
  cada core, en función de los diferentes parámetros de funcionamiento
  que intercom-TM posee. Esto permitirá conocer cómo de bueno es el
  balanceo de la carga entre cores.
\end{enumerate}

\section{Fases de desarrollo}
Se preveen las siguientes unidades de trabajo y temporizaciones (véase
la Figura~\ref{fig:temporizacion}):
\begin{enumerate}
  \item {IDENTIFICACIÓN}: Identificación y comprensión de los
    requerimientos y objetivos planteados (10 h).
  \item {ANÁLISIS:} Estudio y análisis de la implementación actual de
    intercom-TM, comprendiendo los algoritmos que utiliza (60 h).
  \item {APROVISIONAMIENTO}: Instalación y configuración del entorno
    de desarrollo y ejecución: Linux y Python (10h).
  \item {IMPLEMENTACIÓN}: Implementación de la versión paralela de
    intercom-TM (60h).
  \item {EVALUACIÓN}: Evaluación de la solución y cuantificación del
    grado de consecución de los objetivos (30h).
  \item {REDACCIÓN}: Redacción de la memoria (30h).
\end{enumerate}

\begin{figure}
  \begin{center}
    \resizebox{\columnwidth}{!}{
      \begin{ganttchart}{1}{25}{10}
        \gantttitle{Bloques de 10 horas}{25} \\
        \gantttitlelist{1,...,25}{1} \\
        \ganttbar{IDENTIFICACIÓN}{1}{2} \\ % 50h
        \ganttlinkedbar{ANÁLISIS}{3}{9} \\ % 50 h
        \ganttlinkedbar{APROVISIONAMIENTO}{10}{11} \\ % 50 h
        \ganttlinkedbar{IMPLEMENTACIÓN}{12}{18} \\ % 50 h
        \ganttlinkedbar{EVALUACIÓN}{19}{22} \\ % 50 h
        \ganttlinkedbar{REDACCIÓN}{23}{25} % 50 h
      \end{ganttchart}
    }
  \end{center}
  \caption{Temporización del TFG.\label{fig:temporizacion}}
\end{figure}

\section{Requerimientos}
Los requerimientos de la versión paralela de intercom-TM son:

\begin{itemize}
\item Generales:
  \begin{enumerate}
  \item El estilo de programación debe seguir la norma
    PEP-8~\cite{PEP8}.
  \item El idioma usado, tanto para la definición de objetos, métodos,
    variables, etc., como para los comentarios debe ser inglés.
  \item Se usará el paragigma de programación orientado a
    objetos~\cite{schach2008object} para ir obteniendo diferences
    aproximaciones sucesivas al producto final.
  \item El producto debe ejecutarse con éxito al menos en Linux,
    Windows y OSX.
  \end{enumerate}
\item Específicos:
  \begin{enumerate}
  \item Deberán aprovecharse tantos cores como sea posible (la
    aplicación corra con éxito).
  \item Mediante experimentación, deberá comprobarse que la versión
    paralela genera una carga máxima en los cores inferior a la carga
    que genera la versión secuencial, para un determinado conjunto
    (que puede cambiar) de parámetros de funcionamiento.
  \end{enumerate}
\end{itemize}

\section{Resultados esperados}
Los requerimientos (en definitiva, los objetivos) planteados
anteriormente se vefificarán mediante experimentación.  Se definirán
diferentes configuraciones típicas (por ejemplo, (1) transmisión de
audio mono y baja tasa de muestreo de 8000 Hz, (2) tranmisión estereo
y alta tasa de muestreo 48000 Hz) y se comprobará que los
requerimientos específicos se consiguen en la solución encontrada.

\section{Conclusiones esperadas}
Básicamente, se tratará de concluir si el modelo de paralelismo de
datos proporcionado por Python a través de su API
multiprocessing~\cite{multiprocessing} es adacuado (o no) para
conseguir los objetivos definidos (véase la
Sección~\ref{sec:objetivos}).

\bibliographystyle{plain}
\bibliography{../bibliografia}

\begin{center}
  \textbf{Firma del director (codirector)}
\end{center}

\end{document}
